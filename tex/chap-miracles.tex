\chapter{Considering Miracles}\label{ch:miracles}
%!TEX root = main.tex

This chapter explores the concept of {\em miracles}, and how probability can be used to describe them and their evidential support.  We defer a discussion of Christianity's foundational miracle, the Resurrection of Jesus, to Chapter~\ref{ch:resurrection}

\section{How to Approach Miracles}

I listened to a
\href{http://donjohnsonministries.org/discussion-with-naturalist-matthew-ferguson-part-1/}{recent
interview}
(\href{http://donjohnsonministries.org/discussion-with-naturalist-matthew-ferguson-part-2/)}{Part
2}) with Matthew Ferguson on the
\href{http://donjohnsonministries.org/}{Don Johnson show}, which I found
pretty impressive. Matthew Ferguson has a
\href{http://adversusapologetica.wordpress.com/}{very interesting blog}
that I just found, and have been enjoying reading.

\subsection{Initial Comments}\label{pandoc-initial-comments}

However, I did find several points in the informal debate that I thought
could be handled better (from my armchair, of course!). Just to note
that although I think that if I were there, I might have been able to
deal with some of the questions better, I also think that nearly all of
the debate was handled much better than I could have done. What struck
me at one point, in
\href{http://donjohnsonministries.org/discussion-with-naturalist-matthew-ferguson-part-2/)}{Part
2}, was Don's zeal for the
\href{http://www.amazon.com/Miracles-Credibility-New-Testament-Accounts/dp/0801039525}{miracles
collection by Craig Keener}
(\href{http://www.uncrediblehallq.net/2012/01/0shou5/review-of-craig-keeners-miracles/)}{review
of Keener's book here}). He seemed to think that because there were
hundreds of thousands of miracle reports, that that was evidence for
their truth. He was, however, quick to dismiss any comparison with other
pseudosciences. Ferguson admits on his blog that the ``debate came off
as a little ambushy'' on this point, because he hadn't read this book,
and clearly couldn't respond to all of them, but I think that misses the
point. I think one can address the miracle claims without being entirely
dismissive (and sounding closed minded) but putting them in their proper
context.

\subsection{Evaluating Miracle Claims - Some Lessons from
UFOs}\label{pandoc-evaluating-miracle-claims---some-lessons-from-ufos}

So in Keener's book, there is a \textbf{huge} collection of claims of
miracles. We could find an equally large collection of UFO sightings.
Now, Don and other Christians would be quick to dismiss UFO sightings as
irrelevant, but I would raise the questions:

\begin{quote}
\begin{itemize}
\itemsep1pt\parskip0pt\parsep0pt
\item
  Given a set of claims, how do we determine whether they are true?
\item
  Are any of them true?
\item
  Do the number of claims contribute to their truth value?
\end{itemize}
\end{quote}

I believe that the methods we use to determine the veracity of UFO
claims can be used to investigate any claims, remarkable or not,
including miracle claims. To start, we clearly we can't personally
investigate every single claim, and thus cannot comment on ones we
haven't investigated except to note where it seems similar to ones that
we have. I have a friend who I managed (over several years) to break of
his UFO enthusiasm - he was convinced by all of these television shows
claiming evidence for alien spacecraft observations and visitations. He
invited me over to his house periodically to watch these shows to get my
reaction. This is the process that I would use:

\begin{enumerate}
\def\labelenumi{\arabic{enumi}.}
\itemsep1pt\parskip0pt\parsep0pt
\item
  I would write down each specific claim - what was \emph{actually}
  being claimed, and what details were there? (names of places, time,
  who saw what, etc\ldots{})
\item
  I would note any initial inconsistencies (for example, there was once
  where, in the interview process, the different witnesses actually
  described \emph{different} things! this seemed to go unnoticed by the
  reporter)
\item
  I would go home, and try to find out as much about the \emph{original}
  details of the events. It would take me probably at least an hour for
  each case, and some I couldn't track down. However, many of them I
  could. I would read the claims again, and the skeptical accounts, and
  the responses to the skeptics. I would try to see what the actual data
  was, how it was collected, when it was reported, etc\ldots{}
\end{enumerate}

What I found for \emph{every} case that I personally investigated was
the following:

\begin{enumerate}
\def\labelenumi{\arabic{enumi}.}
\itemsep1pt\parskip0pt\parsep0pt
\item
  Most of the \emph{actual}, original claims were mundane. Lights in the
  sky, marks on the ground, etc\ldots{}. No hard evidence of anything
  remarkable.
\item
  Misinterpretation of a known object, or objects, in the sky or on the
  ground.
\item
  The reporting of the claims grew more and more remarkable. A
  particularly good example was the
  \href{http://en.wikipedia.org/wiki/Rendlesham_Forest_incident}{Rendelsham
  Forest} UFO case where the initial reports were just lights, and the
  later reports involved spacecraft, alien code-books, etc\ldots{}
\item
  There were serious inconsistencies between reports, or anomalous
  non-reports (i.e.~people who \emph{should} have seen something but
  didn't). A good example of this was a
  \href{http://en.wikipedia.org/wiki/2006_O'Hare_International_Airport_UFO_sighting}{Chicago
  airport sighting} where a small group of people, in a localized area
  of the airport, saw something yet the large number of other people in
  the nearby areas of the airport reported nothing.
\end{enumerate}

I repeat - in \emph{every single case} that I personally investigated,
these points were in evidence. Then I look through something like the
\href{http://files.ncas.org/condon/text/contents.htm}{Condon report}
where they go through something like 30 years of data in the height of
the UFO craze and don't come up with even a single item that is not
mundane in its nature. After that, new UFO claims I see with suspicion
even if I don't check them out. If something seems straightforward to
check out, I might do it, but I don't feel it is my job to investigate
every claim. If there had been even a single case which pointed to
something probably remarkable, I'd have a different attitude.

\begin{quote}
Lesson: if the claims made shrink and disappear at critical and
skeptical investigation, the claim is not likely to be true.
\end{quote}

\subsection{Miracles}\label{pandoc-miracles}

The Catholic Church has a division to investigate miracles, and has
determined that some of them are genuine. However, the Catholic Church
often has significant blinders, and definitely takes a long time to
adjust to obvious mistakes (Galileo anyone?).

Take, for example,
\href{http://listverse.com/2008/07/14/top-10-astonishing-miracles/}{this
site on top 10 miracles}. I've personally researched about 3 or 4 of
these, and it is quite clear that those are definitely frauds (\#1, 2,
3, and 5 I've checked). Yet, do we get any retraction from the Catholic
Church? Do we get \emph{any} hint of skepticism? None at all.

Again, I follow the same steps as above. I do not take someone else's
word, necessarily, and I don't discount them out of hand. The miracles
of Fatima are a great example. First, we have ``visions'' from highly
impressionable children, one of whom was known to have made up fanciful
stories in the recent past. These children are the only ones who ``see''
it, until the last vision where hundreds claimed to see the ``Miracle of
the Sun''. The problem? The initial stories did not agree, and we only
get a semi-consistent story after the various witnesses spoke with each
other and to a priest collecting the reports. Check out
\href{http://www.csicop.org/si/show/real_secrets_of_fatima/}{The Real
Secrets of Fatima} for the details. All of the elements spoken about
above can be seen - initial mundane experiences, misinterpretation of
known objects (i.e.~the sun, and clouds), the exaggeration of stories in
later recollection, serious inconsistencies in reports and notable
non-reports.

The same goes for every faith-healer I've read about. A little digging,
and a little skepticism, and the entire enterprise come crashing down.
Many times it doesn't take much digging!

\begin{quote}
If the truth is there, then it shouldn't retreat under investigation.
\end{quote}

This is not a matter of being \emph{too skeptical}. It is a matter of
not being credulous.



\section{Healing Miracles}

\subsection{Unbelievable? 17 Nov 2007 - Are miracles evidence for God? -
17 November 2007 -- Miracles and healing - is it evidence for the truth
of
Christianity?}\label{unbelievable-17-nov-2007-are-miracles-evidence-for-god-17-november-2007-miracles-and-healing-is-it-evidence-for-the-truth-of-christianity}

As part of the
\href{https://brianblais.wordpress.com/2013/02/27/unbelievable-project-a-non-believers-armchair-perspective-on-six-years-of-christian-debates/}{Unbelievable
Project}, I am taking notes and ``arm-chair'' responding to each of the
\href{http://www.premierradio.org.uk/shows/saturday/unbelievable.aspx}{Unbelievable
podcast} episodes satisfying a set of
\href{https://brianblais.wordpress.com/2013/02/27/unbelievable-project-a-non-believers-armchair-perspective-on-six-years-of-christian-debates/}{simple
rules}.

For a full RSS Feed of the podcasts
\href{http://ondemand.premier.org.uk/unbelievable/AudioFeed.aspx}{see
here}.

\subsubsection{Description of Episode}\label{description-of-episode}

\begin{itemize}
\item
  Full Title: \emph{Unbelievable? 17 Nov 2007 - Are miracles evidence
  for God? - 17 November 2007 -- Miracles and healing - is it evidence
  for the truth of Christianity?}

  \begin{quote}
  Agnostic sceptic Stephen Pilcher believes that Christian claims to
  healing are tricks of the mind.~ Can John Ryeland of the Christian
  Healing Mission persuade him differently?~ Also features personal
  stories from people who claim to have been miraculously healed.
  \end{quote}
\end{itemize}

\href{http://media.premier.org.uk/unbelievable/ba5b6360-edf3-4218-8878-3292237289f5.mp3}{Download
mp3.}

\begin{itemize}
\itemsep1pt\parskip0pt\parsep0pt
\item
  Justin Brierley - Christian Moderator
\item
  John Ryeland - Christian
\item
  Heather Riley - Christian
\item
  Stephen Pilcher - Agnostic
\end{itemize}

\subsubsection{Notes}\label{notes}

Stephen - \emph{``I'm a church-going, Bible-reading agnostic''}

\textbf{Me - That's pretty funny, because it is very close to what I am.
I am a church-going, Bible-reading atheist. Although I have read the
Bible at least once cover-to-cover a few years ago, I have lost patience
with reading it now. There is so much repetitious and tedious material,
both Old and New Testament, that I find it hard to read for long without
thinking I have better things to do with my time.\\}

Stephen - There are a large number of \emph{``miracles''} that aren't
miracles at all, and non-Christians can have healings as well.

Heather - Her story is at 24:40, in case you want to listen to the
original. Here is a quick summary. She pursued a psychology degree,
studied the paranormal. About 3 or 4 months ago, she went to a
chiropodist (aka podiatrist) who told her that one leg was longer than
another (by an inch). She then went, unplanned, to a religious
gathering. During the meeting the preacher gave her some very specific
information about he from several years ago, some comforting words, and
a moving message. And then he healed the asymmetric leg. She said she
felt like she was \emph{``on show''} and that \emph{``God's really got
to do something''}. She also said that she didn't feel any different,
but a friend of hers observing saw the leg lengthening. So then she went
back to the chiropodist. She determined he was Christian, told the
entire story, he repeated the measurement and then did some more
\emph{robust} measurements, and found no difference in leg length. And
since then, her shoes are no longer asymmetric.

Stephen - People have done studies of faith healings and always come up
short.

\textbf{Me - When I first heard that story, a year or so ago when I
first listened to this episode, I recall being pretty impressed with the
healing. Since then, after much reading, and hearing this again I am
not. (It is interesting that I recall it being more impressive, and if I
never heard the story again, might have started to spread a more
impressive story if I told it again. This is a nice reminder of how
these stories, working with the limitations of memory, can grow in the
telling and quickly become unfactual).\\}

**\\Anyway, why am I not impressed? There are a number of little details
that she dropped in that I find curious. Consider two models (there are
probably more!):**

\begin{enumerate}
\def\labelenumi{\arabic{enumi}.}
\itemsep1pt\parskip0pt\parsep0pt
\item
  \textbf{There is a God, and he decides to heal this leg, but not other
  ailments, and not her husbands problems. This is hard to reconcile
  even on the face of it, and later in the show she talks about this
  somewhat.}
\item
  \textbf{There is no God, these things don't happen, and there are
  other mundane explanations}
\end{enumerate}

**\\Turns out that leg-lengthening is a very common form of
\emph{``healing''} in these sorts of situations (see ``The Faith
Healers'' by James Randi), because it looks impressive and is a
straightforward trick. That's why it is important to have magicians as
well as scientists investigate such claims, because scientists are
terrible at detecting dishonesty and trickery. The fact that she had no
idea that one leg was longer, until a few days before, that she did not
feel the healing, and only went on the word of the friend because she
was expecting something to happen, that she was impressed with the
\emph{``prophecy''} that the preacher said, referring to things he would
have no idea about. James Randi speaks about this at length, and shows
how preachers will use planted people, microphones, and other techniques
to appear to know things they wouldn't already know. Even if they aren't
being deceptive, they may hear in conversations with the friends ahead
of time about Heather's problems, and then work it into the
\emph{``prophecy''}. When she goes back to the chiropodist she finds out
he is a Christian, and \emph{before} he redoes the measurement she tells
the story. Now, this chiropodist has a vested interest in confirming the
healing, because it will confirm his worldview. This is why we have
double-blind measurements, because we \emph{know} people bias the
measurement, the reporting, the memory of it because of their worldview.
In fact, she tells that the chiropodist had to do more robust
measurements to confirm the equal leg length. Perhaps there was an error
in the first measurement. Perhaps the first measurement was
overestimated. Perhaps it wasn't, and \emph{she} reported it rounding up
(i.e.~he says a bit more than 1/2 inch, she tells her friends around an
inch, and then remembers it as such, etc\ldots{}). Perhaps the equipment
for the first test has a bias, which might have motivated him to make
the more robust measurements. There are many possibilities that do not
require dishonesty, deliberate deception, incompetence, and are
completely mundane.**

\textbf{So which model can explain each of these? It seems clear to me
that there are perfectly good mundane explanations for nearly every
detail of the story, that the story is inconsistent even with a
\emph{``real''} healing, and that model 2 is definitely better. What
about her asymmetric shoes, and the pain that occured and went away
after a while after the \emph{``healing''}? My shoes tend to get
asymmetric over time (not with each other, but each pushed off to the
outside) and when I get new shoes, and they are flat, I have a little
pain walking and running that goes away as I adjust. Notice that these
events happened \emph{3 or 4 months ago}. There is no way that her new
shoes would have become asymmetric in that time \emph{anyway}.\\}

John - \emph{``There is an awful lot of anecdotal evidence, and I don't
want to be skeptical of it simply because it is not documented. What
else\ldots{}.is she not telling the truth? Of course she's telling the
truth!''}

\textbf{Me - All we need is a slightly overzealous preacher, a slightly
sloppy chiropodist, and a small amount of congitive bias. It really is
that simple, and it doesn't require us to disparage the character of
\emph{anyone} in the story.}

John - \emph{``How high should we set the bar to know that this is a
proper healing story?''}

John - \emph{``For some people they want to make it so hard to call it a
miracle that nothing could ever satisfy that. I want to take Heather's
story, listen to it, and ask `How did it change her'? If it is a story,
told with integrety, seems to have a lasting effect, of course it would
be better if it were documented, but we don't have the ability to get
the documentation.''}

**Me - I listen to this talk about documentation, and about how it's
*``so hard*``, some people are \emph{``so skeptical''} and I have to
think \emph{``cry baby, cry baby, cry baby''}. I even hear the little
whining voice in my head.\\\emph{``People should be more believing of my
miracle claims''}, \emph{``You're being too skeptical''}, etc\ldots{} Of
course, when it comes to \emph{other} people's miracle claims, they are
just as skeptical! It's only the ones that support their worldview that
they consider for special treatment. Sorry, that's not good enough. Even
Heather points this out, saying that she feels that people are more
skeptical of religious claims than claims of the paranormal (which she
saw in her studies of the paranormal). She's noting, in others, the same
thing she is doing with her worldview. I've posted specifically about
\href{https://brianblais.wordpress.com/2012/09/25/naturalistic-bias-presupposing-naturalism/}{this
problem here}.**

**\\In science, say you are trying to publish a paper, and the editor or
reviewer returns it saying that they are not convinced of your
conclusions, you don't go \emph{``Oh, you're being too hard on me, too
skeptical. Getting the documentation for this effect I am claiming
exists is just too hard.''} That is just ridiculous. You find a way to
document it, with careful measurements, and you convince the skeptics if
it is true, or not if it is false. Truth should convince even the
skeptics, especially if you're claiming a large effect.**

**\\Take the Higgs boson, as an example of an unseen entity for which we
only can get indirect inference of its existence. It was proposed
\emph{50 years ago}, and although people may have thought it was likely
to be there, they didn't \emph{believe} it was there until the proper
measurements were done. Measurements which took decades to set up,
required hundreds of people as a team, and has cost billions of dollars,
just to get the documentation for the existence of something which
doesn't even seem to violate physical law. Think about that next time
you hear someone claim that getting documentation for healing is hard,
or that the effect seems to disappear whenever you look into it
carefully, and that is the reason there isn't any evidence for it.**

\textbf{If the claimed effects of so-called faith-healings are real,
they should be \emph{trivial} to demonstrate, document, and convince the
skeptics.}
