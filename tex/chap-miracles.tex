\chapter{Considering Miracles}\label{ch:miracles}
%!TEX root = main.tex

\newcommand{\hrb}[1]{\href{http://web.bryant.edu/~bblais/#1}{#1}}

% maybe unbelievable-project-miracles-and-healing-is-it-evidence-for-the-truth-of-christianity.html}

% maybe \TODO{what do to in the face of serious uncertainty?}  unbelievable-project-john-lennox-debates-science-and-god-has-science-buried-god.html}


\TODO{change the ``I'' to ``we'' here.}

This chapter explores the concept of {\em miracles}, and how probability can be used to describe them and their evidential support.  We defer a discussion of Christianity's foundational miracle, the Resurrection of Jesus, to Chapter~\ref{ch:resurrection}


\section{Introduction to Miracles}

Perhaps the most influential work on miracles is the chapter ``Of Miracles'' in David Hume's larger work on reason\cite{Hume:1748aa}.  It is also one of the most often \emph{misquoted} books as well.  For example, Normal Geisler misquotes Hume\cite{Brierley:2008ab}, attributing the statement ``The evidence for the regular is always greater than that for the rare.'' to Hume.  I think one of the reasons this occurs is that Hume predates probability theory, so he uses verbose explanations where the modern reader can insert a single statement. For example, reading Hume's work on Miracles, it is clear to me that what he intends to say is simply

\pquote{the prior probability for the regular is always greater than the rare}

a bit less pithy, but more correct (although Hume doesn't phrase it quite like that either). This is at least a factual statement - before the data, we should believe the regular over the rare. That doesn't imply that data couldn't convince us of rare events, it only implies that it is more difficult for data to convince us of rare vs regular events (i.e. you need better evidence for the rare).

In an Unbelievable podcast interview with apologist Norman Geisler\cite{Brierley:2008ab}, Geisler uses the first "quote" of Hume above, and then argues that Hume must be wrong because science already accepts singular and rare events - the Big Bang, the origin of life, and macro evolution (this last one is, of course, neither singular nor rare). Geisler then concludes that miracles can exist! However, seen in the light of probability theory, it becomes clear. 

The prior probability of the events is low:
\begin{eqnarray*}
P(\mbox{Big Bang})\ll 1 \\
P(\mbox{Origin of Life})\ll 1 \\
P(\mbox{Macro Evolution})\ll 1 \\
P(\mbox{Miracles})\ll 1
\end{eqnarray*}
but with data, we have something quite different:
\begin{eqnarray*}
P(\mbox{Big Bang}|{\rm data})\sim 1 \\
P(\mbox{Origin of Life}|{\rm data})\sim 1 \\
P(\mbox{Macro Evolution}|{\rm data})\sim 1 \\
P(\mbox{Miracles}|{\rm data})\sim  0
\end{eqnarray*}
where the data are

\bi
\i Big Bang - \href{http://www.talkorigins.org/faqs/astronomy/bigbang.html}{a long list}
\i Origin of Life - a \href{http://en.wikipedia.org/wiki/Abiogenesis}{nice summary} with RNA world, and autocatalysis
\i Macro Evolution - a term not used by biologists, but \href{http://www.talkorigins.org/faqs/comdesc/}{another long list} for evolution
\i Miracles - nothing convincing to the scientific community
\ei

So, although all these things \emph{can} occur, only a few of them \emph{actually} have evidence strong enough to overcome their initial low prior probability.  Hume recognized this for miracles, as was clear from his writings, although it would have been clearer had he had the benefit of probabilistic vocabulary. 

\section{A Critique of Hume}

One primary critique of Hume comes from Stanford Encyclopedia of Philosophy article on miracles\cite{sep-miracles}.  As we will see, this article does very little to modify Hume's conclusions, but we can use it as a good example of how to use probability mathematics to approach a philosophical article.

\subsection{Concepts and Definitions}\label{concepts-and-definitions}

The article begins by discussing one of Hume's definitions of a miracle
as ``a violation of the laws of nature''. From what I can tell, their
main critique is they don't like the connotations of the word ``law'', a
perspective I share - it is a bit loose terminology, with too many
alternate meanings to be the foundation of a well-defined argument.
Their revised definition is the following:

\begin{quote}
A miracle is an event that exceeds the productive power of nature
\end{quote}

Perhaps this is the scientist in me, and why I am not a philosopher, but
I don't see a striking difference between these two definitions in at
least how they are used. So it seems reasonable to adopt this as a good
working definition.

They go on to clarify a subset of miracle,

\begin{quote}
a religiously significant miracle is a detectable miracle that has a
supernatural cause.
\end{quote}

This clarification is to deal with the following problem, and I'd agree
with at least the sentiment.

\begin{quote}
An insignificant shift in a few grains of sand in the lonesome desert
might, if it exceeded the productive powers of nature, qualify as a
miracle in some thin sense, but it would manifestly lack religious
significance and count not be used as the fulcrum for any interesting
argument.
\end{quote}

I am not sure how, in practice, one would be able to determine a
``supernatural cause'', let alone establish how an event could be beyond
the ``productive power of nature'' without committing a fallacy of
\emph{argument from ignorance}, but let's leave that for now.

\subsection{Arguments for Miracle
Claims}\label{arguments-for-miracle-claims}

This section starts with a quick list of the types of evidence and
arguments made for miracles.

\begin{quote}
Many arguments for miracles adduce the testimony of sincere and able
eyewitnesses as the key piece of evidence on which the force of the
argument depends. But other factors are also cited in favor of miracle
claims: the existence of commemorative ceremonies from earliest times,
for example, or the transformation of the eyewitnesses from fearful
cowards into defiant proclaimers of the resurrection, or the conversion
of St.~Paul, or the growth of the early church under extremely adverse
conditions and without any of the normal conditions of success such as
wealth, patronage, or the use of force. These considerations are often
used jointly in a cumulative argument. It is therefore difficult to
isolate a single canonical argument for most miracle claims. The various
arguments must be handled on a case-by-case basis.
\end{quote}

All of these pieces of so-called evidence are the worst kind of
evidence, for which there are countless examples of the same, or similar
evidence use to shore up the claims of other (presumably false) beliefs.
You can think ``Mormonism'' or ``Alien Abductions'' for nearly every
point listed.

They then outline two types of inductive arguments:

\begin{quote}
\begin{enumerate}
\def\labelenumi{\arabic{enumi}.}
\itemsep1pt\parskip0pt\parsep0pt
\item
  the conclusion (in this case that the miracle in question has actually
  occurred) is probable to some specific degree, or at least more
  probable than not
\item
  the conclusion is more probable given the evidence presented than it
  is considered independently of that evidence
\end{enumerate}
\end{quote}

Point (1) is just either specifying either
\(P({\rm miracle}|{\rm data})\) directly or establishing only that
\(P({\rm miracle}|{\rm data})>0.5\). Point (2) is
\(P({\rm miracle}|{\rm data})>P({\rm miracle})\). Point (2) is nearly
useless. For example, you could have

\begin{eqnarray*}
P({\rm miracle})&=&0.00001 \\
P({\rm miracle}|{\rm data})&=&0.001
\end{eqnarray*}

and still have a seriously unlikely hypothesis, even given a factor of
100 increase in probability of a miracle given the data. Thus the
\emph{only} thing that matters must be the actual value of
\(P({\rm miracle}|{\rm data})\).

One such argument for miracles specifies the type of evidence needed to
make one confident that one is talking about a miracle. The article
calls this a ``criteriological'' argument, but all of the arguments
dealt with are probabilistic. What are the criteria, for example? This
one is from Charles Leslie:

\begin{quote}
\begin{enumerate}
\def\labelenumi{\arabic{enumi}.}
\itemsep1pt\parskip0pt\parsep0pt
\item
  That the matters of fact be such, as that men's outward senses, their
  eyes and ears, may be judges of it.
\item
  That it be done publicly in the face of the world.
\item
  That not only public monuments be kept up in memory of it, but some
  outward actions to be performed.
\item
  That such monuments, and such actions or observances, be instituted,
  and do commence from the time that the matter of fact was done.
\end{enumerate}
\end{quote}

One can easily site both the golden plates of Joseph Smith, and also the
events surrounding Roswell, that satisfy all of these. Clearly, there is
an issue with them.

Another common argument is called the ``minimal facts'' approach. The
best summary, and take-down of this argument is on
\href{https://adversusapologetica.wordpress.com/2013/06/29/knocking-out-the-pillars-of-the-minimal-facts-apologetic/}{Matthew
Ferguson's blog}. One essential missing part of the minimal facts
approach is that it only includes \emph{likelihoods} and not
\emph{priors}, and thus fails a basic probabilistic analysis.

\subsubsection{Probabilistic arguments}\label{probabilistic-arguments}

The first form here deals with \emph{testimony}, with the following
assumptions and conventions:

\begin{enumerate}
\def\labelenumi{\arabic{enumi}.}
\itemsep1pt\parskip0pt\parsep0pt
\item
  \(T_i\equiv\) the proposition ``Witness \(i\) testifies to \(M\)''
\item
  \(P(T_i,T_j) = P(T_i)\times P(T_j)\): independence
\item
  \(P(T_i|M)=P(T_j|M)\) for all \(i\) and \(j\): all testimony is
  equally informative
\end{enumerate}

We then easily derive: \[
\frac{P(M|T_1,T_2,\cdots,T_n)}{P(\sim\!M|T_1,T_2,\cdots,T_n)} = \left(\frac{P(T_1|M)}{P(T_1|\sim\!M)}\right)^n \times \frac{P(M)}{P(\sim\!M)}
\]

The article then spins this in a positive way toward miracles:
\textgreater{}{[}I{]}f independent witnesses can be found, who speak
truth more frequently than falsehood, \emph{it is ALWAYS possible to
assign a number of independent witnesses, the improbability of the
falsehood of whose concurring testimony shall be greater than the
improbability of the alleged miracle.} (Babbage 1837: 202, emphasis
original; cf.~Holder 1998 and Earman 2000)

However, comparing with Hume, it becomes obvious why this spin fails:
\textgreater{} When anyone tells me, that he saw a dead man restored to
life, I immediately consider with myself, whether it be more probable,
that this person should either deceive or be deceived, or that the fact,
which he relates, should have really happened. I weigh the one miracle
against the other; and according to the superiority, which I discover, I
pronounce my decision, and always reject the greater miracle. If the
falsehood of the testimony would be more miraculous, than the event
which he relates; then, and not till then, can he pretend to command my
belief or opinion. (Hume)

The first quote implies that the terms \(P(T_1|M)\) and
\(P(T_1|\sim\!M)\) refer to speaking truth vs falsehoods (i.e.~lying),
as opposed to speaking correctly vs being mistaken. In the latter, it is
very easy to see why, for certain types of extraordinary events, we
would expect fallible observers to have \(P(T_1|\sim\!M)>P(T_1|M)\) and
further that even \emph{if} witnesses were in general slightly more
reliable than not, we can't expect the observations to be
\emph{independent} in general. In the specific case of the (anonymous)
Gospel writers, there is strong evidence of \emph{dependence} in the
accounts to make this entire calculation (except in its gross
qualitative features) irrelevant.

\subsection{Arguments against miracles}\label{arguments-against-miracles}

Quoting Hume again,

\begin{quote}
The plain consequence is (and it is a general maxim worthy of our
attention), ``That no testimony is sufficient to establish a miracle,
unless the testimony be of such a kind, that its falsehood would be more
miraculous, than the fact, which it endeavors to establish: And even in
that case, there is a mutual destruction of arguments, and the superior
only gives us an assurance suitable to that degree of force, which
remains, after deducting the inferior.''
\end{quote}

This is correct, and is a direct statement of Bayesian reasoning \[
\frac{P(M|E)}{P(\sim\!M|E)} =\frac{P(E|M)}{P(E|\sim\!M)} \times\frac{P(M)}{P(\sim\!M)}
\] where we can use the approximations \(P(E|M)\approx 1\) and
\(P(\sim\!M)\approx 1\) and achieve \[
\frac{P(M|E)}{P(\sim\!M|E)} \approx \frac{P(M)}{P(E|\sim\!M)}
\]

The \href{http://plato.stanford.edu/entries/miracles/}{article on
miracles} continues to try to map this to a philosophical structure
(needlessly, I'd say), with the following ``simple version'' of the
argument:

\begin{quote}
A very simple version of the argument, leaving out the comparison to the
laws of nature and focusing on the alleged infirmities of testimony, can
be laid out deductively (following Whately, in Paley 1859: 33):

\begin{enumerate}
\def\labelenumi{\arabic{enumi})}
\item
  Testimony is a kind of evidence very likely to be false.
\item
  The evidence for the Christian miracles is testimony.
\end{enumerate}

Therefore,

\begin{enumerate}
\def\labelenumi{\arabic{enumi})}
\setcounter{enumi}{2}
\itemsep1pt\parskip0pt\parsep0pt
\item
  The evidence for the Christian miracles is likely to be false.
\end{enumerate}

This is, however, much too crude an argument to carry any weight, since
it turns on a simple ambiguity between all testimony and some testimony.
Whately offers an amusing parody that makes the fallacy obvious: Some
books are mere trash; Hume's Works are {[}some{]} books; therefore, etc.
\end{quote}

It's remarkable that such a silly parallel is seriously made. The
structure isn't really parallel at all, so let's make it explicit:

\begin{enumerate}
\def\labelenumi{\arabic{enumi}.}
\itemsep1pt\parskip0pt\parsep0pt
\item
  Books are likely to be trash. (in other words, most books are trash)
\item
  Hume wrote some books
\item
  therefore, Hume's books are likely to be trash.
\end{enumerate}

This is a correct argument, given the premises. If all we knew was that
``some guy named Hume'' wrote ``some book'', then with all probability
(if premise 1. is correct) that book would be trash. The issue is that,
unconsciously, we are inserting extra information - Hume was a famous
philosopher, he had a particular education, etc\ldots{} With this extra
information, we would have a hard time supporting a similar premise as
1. above.

The fact that this is so trivially dispensed with makes one wonder - why
would anyone be convinced by this? Why couldn't the author of the
article see it? It smacks of grasping at straws to try to dispel Hume's
main arguments.

The article continues with some odd re-phrasings of Hume, where the
mathematics is just the single line above. I don't understand all the
work. A strange one is then critiqued with an even stranger statement:
\textgreater{}The presumptive case against the resurrection from
universal testimony would be as strong as Hume supposes only if,
\emph{per impossible}, all mankind throughout all ages had been watching
the tomb of Jesus on the morning of the third day and testified that
nothing occurred. Even aside from the problems of time travel, there is
not a \emph{single piece} of direct testimonial evidence to Jesus'
non-resurrection.

Does anyone seriously think that the case against a claim always (or
even usually) takes the form of direct testimony against that claim?
Where is the testimony that Zeus didn't exist? Anyone who can explain
this odd line of reasoning, please chime in.

\subsection{Particular Arguments}\label{particular-arguments}

According to the article, Hume lists a set of conditions needed to make
testimony carry maximum weight:

\begin{quote}
{[}T{]}here is not to be found, in all history, any miracle attested by
a sufficient number of men, of such unquestioned good sense, education,
and learning, as to secure us against all delusion in themselves; of
such undoubted integrity, as to place them beyond all suspicion of any
design to deceive others; of such credit and reputation in the eyes of
mankind, as to have a great deal to lose in case of their being detected
in any falsehood; and at the same time attesting facts, performed in
such a public manner, and in so celebrated a part of the world, as to
render the detection unavoidable: All which circumstances are requisite
to give us a full assurance in the testimony of men. (Hume 1748/2000:
88)
\end{quote}

Essentially, he is saying that the methods of science have never
confirmed a miracle. The methods of science help ``secure us against all
delusion in themselves'', remove ``suspicion of any design to deceive
others'', with processes ``performed in a public manner'' that ``render
the detection unavoidable''.

It is criticized by noting that some of these conditions can cut the
other way, such as the condition of ``credit and reputation'',

\begin{quote}
It might have been said with some shew of plausibility, that such
persons by their knowledge and abilities, their reputation and interest,
might have it in their power to countenance and propagate an imposture
among the people, and give it some credit in the world. (Leland 1755:
90--91; cf.~Beckett 1883: 29--37)
\end{quote}

This is, essentially, pointing out fallacy of authority - a good
critique. Science, by its processes, attempts to avoid that as well. Of
course, Hume predates modern science, so I think we can forgive him some
sloppiness or poor choice of terminology.

Hume continues to suggest that the religious context of the miracle
claime makes the telling of the miracle story even more likely. This
would increase the probability of obtaining the testimony even if no
miracle happened - \(P(E|\sim\!M)\) increases - making the probability
of a miracle go down. The criticism here? The effect could happen in the
other direction:

\begin{quote}
But as George Campbell points out (1762/1839: 48--49), this
consideration cuts both ways; the religious nature of the claim may also
operate to make it less readily received:

\begin{quote}
{[}T{]}he prejudice resulting from the religious affection, may just as
readily obstruct as promote our faith in a religious miracle. What
things in nature are more contrary, than one religion is to another
religion? They are just as contrary as light and darkness, truth and
error. The affections with which they are contemplated by the same
person, are just as opposite as desire and aversion, love and hatred.
The same religious zeal which gives the mind of a Christian a propensity
to the belief of a miracle in support of Christianity, will inspire him
with an aversion from the belief of a miracle in support of
Mahometanism. The same principle which will make him acquiesce in
evidence less than sufficient in one case, will make him require
evidence more than sufficient in the other\ldots{}.
\end{quote}
\end{quote}

I disagree quite strongly with this line of thinking. One of the big
problems with pseudoscience is that it promotes poor thinking in other
domains. Someone who believes in miracles will not find it hard to
believe that the miracle claims of other religions are at least
possible. If you believe in unseen agents, then to move from
Christianity to New Age to Scientology isn't that large of a stretch.
Often, when ones religion is undermined, the typical response is to
switch to another religion! Thus, they are not as opposite as ``light
and darkness''. Poor thinking is poor thinking, regardless of the
context.

\subsubsection{Argument from Parity}\label{argument-from-parity}

Hume brings up miracles in other religions. In a fit of special
pleading, the \href{http://plato.stanford.edu/entries/miracles/}{article
on miracles} retorts,

\begin{quote}
All attempts to draw an evidential parallel between the miracles of the
New Testament and the miracle stories of later ecclesiastical history
are therefore dubious. There are simply more resources for explaining
how the ecclesiastical stories, which were promoted to an established
and favorably disposed audience, could have arisen and been believed
without there being any truth to the reports.
\end{quote}

The argument is quite simple - if there are known cases of miracle
claims where no miracle actually occurred, that increase
\(P(E|\sim\!M)\), making the probability of a miracle go down given
testimony. It doesn't matter whether you have good reasons to believe
there was no miracle for these cases - it undermines testimony of
miracles in general.

\subsection{In conclusion}\label{in-conclusion}

So, as far as I can tell, there is no substantive critique to Hume's
statements about miracles. He lacks the rigor of the mathematics of
probability, but his wording is so straightforwardly translated to it
that I find it difficult to see what the problem is. I also found it
ironic that the entire article, which has been pro-miracle the entire
time, ends with this:

\begin{quote}
For the evidence for a miracle claim, being public and empirical, is
never strictly demonstrative, either as to the fact of the event or as
to the supernatural cause of the event. It remains possible, though the
facts in the case may in principle render it wildly improbable, that the
testifier is either a deceiver or himself deceived; and so long as those
possibilities exist, there will be logical space for other forms of
evidence to bear on the conclusion. Arguments about miracles therefore
take their place as one piece---a fascinating piece---in a larger and
more important puzzle.
\end{quote}

This is pretty much exactly what Hume was saying. Given that there is
always a non-zero probability of the testifier either lying or being
mistaken, one has to establish the evidence for a miracle strong enough
to overcome both the negligibly small prior probability and this
non-zero probability of the testimony being wrong. Since mistakes are a
common human trait, and distortions are also common on testimony,
evidence for miracles according to probability theory, Hume, and all
rational thought have always been found lacking.



%========================



\section{How to Approach Miracles}

I listened to a
\href{http://donjohnsonministries.org/discussion-with-naturalist-matthew-ferguson-part-1/}{recent
interview}
(\href{http://donjohnsonministries.org/discussion-with-naturalist-matthew-ferguson-part-2/)}{Part
2}) with Matthew Ferguson on the
\href{http://donjohnsonministries.org/}{Don Johnson show}, which I found
pretty impressive. Matthew Ferguson has a
\href{http://adversusapologetica.wordpress.com/}{very interesting blog}
that I just found, and have been enjoying reading.

\subsection{Initial Comments}\label{pandoc-initial-comments}

However, I did find several points in the informal debate that I thought
could be handled better (from my armchair, of course!). Just to note
that although I think that if I were there, I might have been able to
deal with some of the questions better, I also think that nearly all of
the debate was handled much better than I could have done. What struck
me at one point, in
\href{http://donjohnsonministries.org/discussion-with-naturalist-matthew-ferguson-part-2/)}{Part
2}, was Don's zeal for the
\href{http://www.amazon.com/Miracles-Credibility-New-Testament-Accounts/dp/0801039525}{miracles
collection by Craig Keener}
(\href{http://www.uncrediblehallq.net/2012/01/0shou5/review-of-craig-keeners-miracles/)}{review
of Keener's book here}). He seemed to think that because there were
hundreds of thousands of miracle reports, that that was evidence for
their truth. He was, however, quick to dismiss any comparison with other
pseudosciences. Ferguson admits on his blog that the ``debate came off
as a little ambushy'' on this point, because he hadn't read this book,
and clearly couldn't respond to all of them, but I think that misses the
point. I think one can address the miracle claims without being entirely
dismissive (and sounding closed minded) but putting them in their proper
context.

\subsection{Evaluating Miracle Claims - Some Lessons from
UFOs}\label{pandoc-evaluating-miracle-claims---some-lessons-from-ufos}

So in Keener's book, there is a \textbf{huge} collection of claims of
miracles. We could find an equally large collection of UFO sightings.
Now, Don and other Christians would be quick to dismiss UFO sightings as
irrelevant, but I would raise the questions:

\begin{quote}
\begin{itemize}
\itemsep1pt\parskip0pt\parsep0pt
\item
  Given a set of claims, how do we determine whether they are true?
\item
  Are any of them true?
\item
  Do the number of claims contribute to their truth value?
\end{itemize}
\end{quote}

I believe that the methods we use to determine the veracity of UFO
claims can be used to investigate any claims, remarkable or not,
including miracle claims. To start, we clearly we can't personally
investigate every single claim, and thus cannot comment on ones we
haven't investigated except to note where it seems similar to ones that
we have. I have a friend who I managed (over several years) to break of
his UFO enthusiasm - he was convinced by all of these television shows
claiming evidence for alien spacecraft observations and visitations. He
invited me over to his house periodically to watch these shows to get my
reaction. This is the process that I would use:

\begin{enumerate}
\def\labelenumi{\arabic{enumi}.}
\itemsep1pt\parskip0pt\parsep0pt
\item
  I would write down each specific claim - what was \emph{actually}
  being claimed, and what details were there? (names of places, time,
  who saw what, etc\ldots{})
\item
  I would note any initial inconsistencies (for example, there was once
  where, in the interview process, the different witnesses actually
  described \emph{different} things! this seemed to go unnoticed by the
  reporter)
\item
  I would go home, and try to find out as much about the \emph{original}
  details of the events. It would take me probably at least an hour for
  each case, and some I couldn't track down. However, many of them I
  could. I would read the claims again, and the skeptical accounts, and
  the responses to the skeptics. I would try to see what the actual data
  was, how it was collected, when it was reported, etc\ldots{}
\end{enumerate}

What I found for \emph{every} case that I personally investigated was
the following:

\begin{enumerate}
\def\labelenumi{\arabic{enumi}.}
\itemsep1pt\parskip0pt\parsep0pt
\item
  Most of the \emph{actual}, original claims were mundane. Lights in the
  sky, marks on the ground, etc\ldots{}. No hard evidence of anything
  remarkable.
\item
  Misinterpretation of a known object, or objects, in the sky or on the
  ground.
\item
  The reporting of the claims grew more and more remarkable. A
  particularly good example was the
  \href{http://en.wikipedia.org/wiki/Rendlesham_Forest_incident}{Rendelsham
  Forest} UFO case where the initial reports were just lights, and the
  later reports involved spacecraft, alien code-books, etc\ldots{}
\item
  There were serious inconsistencies between reports, or anomalous
  non-reports (i.e.~people who \emph{should} have seen something but
  didn't). A good example of this was a
  \href{http://en.wikipedia.org/wiki/2006_O'Hare_International_Airport_UFO_sighting}{Chicago
  airport sighting} where a small group of people, in a localized area
  of the airport, saw something yet the large number of other people in
  the nearby areas of the airport reported nothing.
\end{enumerate}

I repeat - in \emph{every single case} that I personally investigated,
these points were in evidence. Then I look through something like the
\href{http://files.ncas.org/condon/text/contents.htm}{Condon report}
where they go through something like 30 years of data in the height of
the UFO craze and don't come up with even a single item that is not
mundane in its nature. After that, new UFO claims I see with suspicion
even if I don't check them out. If something seems straightforward to
check out, I might do it, but I don't feel it is my job to investigate
every claim. If there had been even a single case which pointed to
something probably remarkable, I'd have a different attitude.

\begin{quote}
Lesson: if the claims made shrink and disappear at critical and
skeptical investigation, the claim is not likely to be true.
\end{quote}

\subsection{Miracles}\label{pandoc-miracles}

The Catholic Church has a division to investigate miracles, and has
determined that some of them are genuine. However, the Catholic Church
often has significant blinders, and definitely takes a long time to
adjust to obvious mistakes (Galileo anyone?).

Take, for example,
\href{http://listverse.com/2008/07/14/top-10-astonishing-miracles/}{this
site on top 10 miracles}. I've personally researched about 3 or 4 of
these, and it is quite clear that those are definitely frauds (\#1, 2,
3, and 5 I've checked). Yet, do we get any retraction from the Catholic
Church? Do we get \emph{any} hint of skepticism? None at all.

Again, I follow the same steps as above. I do not take someone else's
word, necessarily, and I don't discount them out of hand. The miracles
of Fatima are a great example. First, we have ``visions'' from highly
impressionable children, one of whom was known to have made up fanciful
stories in the recent past. These children are the only ones who ``see''
it, until the last vision where hundreds claimed to see the ``Miracle of
the Sun''. The problem? The initial stories did not agree, and we only
get a semi-consistent story after the various witnesses spoke with each
other and to a priest collecting the reports. Check out
\href{http://www.csicop.org/si/show/real_secrets_of_fatima/}{The Real
Secrets of Fatima} for the details. All of the elements spoken about
above can be seen - initial mundane experiences, misinterpretation of
known objects (i.e.~the sun, and clouds), the exaggeration of stories in
later recollection, serious inconsistencies in reports and notable
non-reports.

The same goes for every faith-healer I've read about. A little digging,
and a little skepticism, and the entire enterprise come crashing down.
Many times it doesn't take much digging!

\begin{quote}
If the truth is there, then it shouldn't retreat under investigation.
\end{quote}

This is not a matter of being \emph{too skeptical}. It is a matter of
not being credulous.



\section{Healing Miracles}

\TODO{fix the formatting here.}

\subsection{Unbelievable? 17 Nov 2007 - Are miracles evidence for God? -
17 November 2007 -- Miracles and healing - is it evidence for the truth
of
Christianity?}\label{unbelievable-17-nov-2007-are-miracles-evidence-for-god-17-november-2007-miracles-and-healing-is-it-evidence-for-the-truth-of-christianity}

As part of the
\href{https://brianblais.wordpress.com/2013/02/27/unbelievable-project-a-non-believers-armchair-perspective-on-six-years-of-christian-debates/}{Unbelievable
Project}, I am taking notes and ``arm-chair'' responding to each of the
\href{http://www.premierradio.org.uk/shows/saturday/unbelievable.aspx}{Unbelievable
podcast} episodes satisfying a set of
\href{https://brianblais.wordpress.com/2013/02/27/unbelievable-project-a-non-believers-armchair-perspective-on-six-years-of-christian-debates/}{simple
rules}.

For a full RSS Feed of the podcasts
\href{http://ondemand.premier.org.uk/unbelievable/AudioFeed.aspx}{see
here}.

\subsubsection{Description of Episode}\label{description-of-episode}

\begin{itemize}
\item
  Full Title: \emph{Unbelievable? 17 Nov 2007 - Are miracles evidence
  for God? - 17 November 2007 -- Miracles and healing - is it evidence
  for the truth of Christianity?}

  \begin{quote}
  Agnostic sceptic Stephen Pilcher believes that Christian claims to
  healing are tricks of the mind.~ Can John Ryeland of the Christian
  Healing Mission persuade him differently?~ Also features personal
  stories from people who claim to have been miraculously healed.
  \end{quote}
\end{itemize}

\href{http://media.premier.org.uk/unbelievable/ba5b6360-edf3-4218-8878-3292237289f5.mp3}{Download
mp3.}

\begin{itemize}
\itemsep1pt\parskip0pt\parsep0pt
\item
  Justin Brierley - Christian Moderator
\item
  John Ryeland - Christian
\item
  Heather Riley - Christian
\item
  Stephen Pilcher - Agnostic
\end{itemize}

\subsubsection{Notes}\label{notes}

Stephen - \emph{``I'm a church-going, Bible-reading agnostic''}

\textbf{Me - That's pretty funny, because it is very close to what I am.
I am a church-going, Bible-reading atheist. Although I have read the
Bible at least once cover-to-cover a few years ago, I have lost patience
with reading it now. There is so much repetitious and tedious material,
both Old and New Testament, that I find it hard to read for long without
thinking I have better things to do with my time.\\}

Stephen - There are a large number of \emph{``miracles''} that aren't
miracles at all, and non-Christians can have healings as well.

Heather - Her story is at 24:40, in case you want to listen to the
original. Here is a quick summary. She pursued a psychology degree,
studied the paranormal. About 3 or 4 months ago, she went to a
chiropodist (aka podiatrist) who told her that one leg was longer than
another (by an inch). She then went, unplanned, to a religious
gathering. During the meeting the preacher gave her some very specific
information about he from several years ago, some comforting words, and
a moving message. And then he healed the asymmetric leg. She said she
felt like she was \emph{``on show''} and that \emph{``God's really got
to do something''}. She also said that she didn't feel any different,
but a friend of hers observing saw the leg lengthening. So then she went
back to the chiropodist. She determined he was Christian, told the
entire story, he repeated the measurement and then did some more
\emph{robust} measurements, and found no difference in leg length. And
since then, her shoes are no longer asymmetric.

Stephen - People have done studies of faith healings and always come up
short.

\textbf{Me - When I first heard that story, a year or so ago when I
first listened to this episode, I recall being pretty impressed with the
healing. Since then, after much reading, and hearing this again I am
not. (It is interesting that I recall it being more impressive, and if I
never heard the story again, might have started to spread a more
impressive story if I told it again. This is a nice reminder of how
these stories, working with the limitations of memory, can grow in the
telling and quickly become unfactual).\\}

**\\Anyway, why am I not impressed? There are a number of little details
that she dropped in that I find curious. Consider two models (there are
probably more!):**

\begin{enumerate}
\def\labelenumi{\arabic{enumi}.}
\itemsep1pt\parskip0pt\parsep0pt
\item
  \textbf{There is a God, and he decides to heal this leg, but not other
  ailments, and not her husbands problems. This is hard to reconcile
  even on the face of it, and later in the show she talks about this
  somewhat.}
\item
  \textbf{There is no God, these things don't happen, and there are
  other mundane explanations}
\end{enumerate}

**\\Turns out that leg-lengthening is a very common form of
\emph{``healing''} in these sorts of situations (see ``The Faith
Healers'' by James Randi), because it looks impressive and is a
straightforward trick. That's why it is important to have magicians as
well as scientists investigate such claims, because scientists are
terrible at detecting dishonesty and trickery. The fact that she had no
idea that one leg was longer, until a few days before, that she did not
feel the healing, and only went on the word of the friend because she
was expecting something to happen, that she was impressed with the
\emph{``prophecy''} that the preacher said, referring to things he would
have no idea about. James Randi speaks about this at length, and shows
how preachers will use planted people, microphones, and other techniques
to appear to know things they wouldn't already know. Even if they aren't
being deceptive, they may hear in conversations with the friends ahead
of time about Heather's problems, and then work it into the
\emph{``prophecy''}. When she goes back to the chiropodist she finds out
he is a Christian, and \emph{before} he redoes the measurement she tells
the story. Now, this chiropodist has a vested interest in confirming the
healing, because it will confirm his worldview. This is why we have
double-blind measurements, because we \emph{know} people bias the
measurement, the reporting, the memory of it because of their worldview.
In fact, she tells that the chiropodist had to do more robust
measurements to confirm the equal leg length. Perhaps there was an error
in the first measurement. Perhaps the first measurement was
overestimated. Perhaps it wasn't, and \emph{she} reported it rounding up
(i.e.~he says a bit more than 1/2 inch, she tells her friends around an
inch, and then remembers it as such, etc\ldots{}). Perhaps the equipment
for the first test has a bias, which might have motivated him to make
the more robust measurements. There are many possibilities that do not
require dishonesty, deliberate deception, incompetence, and are
completely mundane.**

\textbf{So which model can explain each of these? It seems clear to me
that there are perfectly good mundane explanations for nearly every
detail of the story, that the story is inconsistent even with a
\emph{``real''} healing, and that model 2 is definitely better. What
about her asymmetric shoes, and the pain that occured and went away
after a while after the \emph{``healing''}? My shoes tend to get
asymmetric over time (not with each other, but each pushed off to the
outside) and when I get new shoes, and they are flat, I have a little
pain walking and running that goes away as I adjust. Notice that these
events happened \emph{3 or 4 months ago}. There is no way that her new
shoes would have become asymmetric in that time \emph{anyway}.\\}

John - \emph{``There is an awful lot of anecdotal evidence, and I don't
want to be skeptical of it simply because it is not documented. What
else\ldots{}.is she not telling the truth? Of course she's telling the
truth!''}

\textbf{Me - All we need is a slightly overzealous preacher, a slightly
sloppy chiropodist, and a small amount of congitive bias. It really is
that simple, and it doesn't require us to disparage the character of
\emph{anyone} in the story.}

John - \emph{``How high should we set the bar to know that this is a
proper healing story?''}

John - \emph{``For some people they want to make it so hard to call it a
miracle that nothing could ever satisfy that. I want to take Heather's
story, listen to it, and ask `How did it change her'? If it is a story,
told with integrety, seems to have a lasting effect, of course it would
be better if it were documented, but we don't have the ability to get
the documentation.''}

**Me - I listen to this talk about documentation, and about how it's
*``so hard*``, some people are \emph{``so skeptical''} and I have to
think \emph{``cry baby, cry baby, cry baby''}. I even hear the little
whining voice in my head.\\\emph{``People should be more believing of my
miracle claims''}, \emph{``You're being too skeptical''}, etc\ldots{} Of
course, when it comes to \emph{other} people's miracle claims, they are
just as skeptical! It's only the ones that support their worldview that
they consider for special treatment. Sorry, that's not good enough. Even
Heather points this out, saying that she feels that people are more
skeptical of religious claims than claims of the paranormal (which she
saw in her studies of the paranormal). She's noting, in others, the same
thing she is doing with her worldview. I've posted specifically about
\href{https://brianblais.wordpress.com/2012/09/25/naturalistic-bias-presupposing-naturalism/}{this
problem here}.**

**\\In science, say you are trying to publish a paper, and the editor or
reviewer returns it saying that they are not convinced of your
conclusions, you don't go \emph{``Oh, you're being too hard on me, too
skeptical. Getting the documentation for this effect I am claiming
exists is just too hard.''} That is just ridiculous. You find a way to
document it, with careful measurements, and you convince the skeptics if
it is true, or not if it is false. Truth should convince even the
skeptics, especially if you're claiming a large effect.**

**\\Take the Higgs boson, as an example of an unseen entity for which we
only can get indirect inference of its existence. It was proposed
\emph{50 years ago}, and although people may have thought it was likely
to be there, they didn't \emph{believe} it was there until the proper
measurements were done. Measurements which took decades to set up,
required hundreds of people as a team, and has cost billions of dollars,
just to get the documentation for the existence of something which
doesn't even seem to violate physical law. Think about that next time
you hear someone claim that getting documentation for healing is hard,
or that the effect seems to disappear whenever you look into it
carefully, and that is the reason there isn't any evidence for it.**

\textbf{If the claimed effects of so-called faith-healings are real,
they should be \emph{trivial} to demonstrate, document, and convince the
skeptics.}
