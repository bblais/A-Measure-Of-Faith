\def\naive{na\"ive}
\def\Naive{Na\"ive}
\def\tightlist{}

%\newcommand{\psy}[2]{\mbox{\psfig{figure=#1,height=#2}}}
%\newcommand{\psx}[2]{\mbox{\psfig{figure=#1,width=#2}}}
\newcommand{\psy}[2]{\includegraphics[height=#2]{#1}}
\newcommand{\psx}[2]{\includegraphics[width=#2]{#1}}

\usepackage{ifthen}

\ifthenelse{\isundefined{\comment}}{
\newcommand{\comment}[1]{}
\newcommand{\uncomment}[1]{#1}
}{}

%\newcommand{\comment}[1]{}
%\newcommand{\uncomment}[1]{#1}


\newcommand{\erf}{{\rm erf}}

\newcommand{\myfootnote}[2]{
\renewcommand{\thefootnote}{#1}
\footnotetext{#2}
\renewcommand{\thefootnote}{\arabic{footnote}}
}

\newcounter{pushpopcounter}
\newcommand{\pushcounter}[1]{
  \setcounter{pushpopcounter}{\value{#1}}
}

\newcommand{\popcounter}[1]{
  \setcounter{#1}{\value{pushpopcounter}}
}


\newcommand{\markmodified}{\myfootnote{$\ast$}{Document last updated: \today}}
\newcommand{\marksupported}{\myfootnote{$\circ$}{This work supported in part by
the Charles A. Dana Foundation and the Office of Naval Research}}

% ------ formatting defs  ------
%
\newcommand{\nn}{\nonumber }
\newcommand{\cc}[1]{\begin{center}#1\end{center}}
\newcommand{\fl}[1]{\begin{flushleft}#1\end{flushleft}}
\newcommand{\fr}[1]{\begin{flushright}#1\end{flushright}}
\newcommand{\fullhline}{\begin{flushleft}\ \hrulefill \ \end{flushleft}}
\newcommand{\todo}[1]{\cc{$\left[ \parbox{5in}{\cc{\large \sc #1}} \right]$}}
\def\xSplitRef#1.#2.#3#4{\setcounter{#3}{#1}\setcounter{#4}{#2}}
\def\SplitRef#1#2#3{\expandafter\xSplitRef\number#1.{#2}{#3}}
%
\newcommand{\obsv}[1]{\begin{itemize}\item #1\end{itemize}}
\newcommand{\E}[1]{\cdot 10^{#1}}
%
\newcommand{\stdtitle}[1]{
\renewcommand{\thefootnote}{\ast}
\footnotetext{Document last updated: \today}
\renewcommand{\thefootnote}{\arabic{footnote}}
\begin{center}{\Large \bf #1}\end{center}
}
%
\newcommand{\centerbox}[1]{\hspace*{\fill}\fbox{#1}\hspace*{\fill}}
%
\newenvironment{boxequation}
{
\begin{equation}\begin{array}{|l|} \hline\raisebox{.3in}{\hbox{}} \displaystyle
}{
\raisebox{-.2in}{\hbox{}} \\ \hline \end{array}\end{equation}
}
%
%
\newenvironment{boxequation*}
{
\[\begin{array}{|l|} \hline\raisebox{.3in}{\hbox{}} \displaystyle
}{
\raisebox{-.2in}{\hbox{}} \\ \hline \end{array}\]
}
%
\def\now{ %
  \count0=\time
  \divide\count0 by 60
  \count2=\count0
  \multiply\count2 by 60
  \count1=\time 
  \advance\count1 by-\count2
  \ifnum\count1<10  \number\count0:0\number\count1
  \else \number\count0:\number\count1
  \fi
}
%
\newcommand{\marginlabel}[1]
  {\mbox{}\marginpar{\raggedleft\hspace{0pt}#1}}
%

\newlength{\overstrikelength}
\newcommand{\overstrike}[1]{
\settowidth{\overstrikelength}{#1}
{#1}\hspace{-\overstrikelength}\parbox{\overstrikelength}{\hrulefill}
}


\newlength{\emptypicturelength}
\newcommand{\emptypicture}[2]{
\setlength{\emptypicturelength}{#1}
\addtolength{\emptypicturelength}{-1.5\emptypicturelength}
\framebox[#2]{\rule[\emptypicturelength]{0in}{#1}Empty Picture}}

%
\newcommand{\temppicture}[3]{
\setlength{\emptypicturelength}{#1}
\addtolength{\emptypicturelength}{-1.5\emptypicturelength}
\framebox[#2]{\rule[\emptypicturelength]{0in}{#1}#3}}

% ------ slide defs  ------
%
\newcommand{\bbslidesize}{\LARGE}
\newenvironment{bbslide}[1]
{\thispagestyle{empty}
\vspace*{\fill}  \bbslidesize \begin{center} {\bf #1} 
\fullhline 
}{
\vspace*{\fill} \pagebreak\end{center}
}
%
\newenvironment{bbuntitledslide}
{\thispagestyle{empty}
\vspace*{\fill} \bbslidesize \begin{center}}{
\vspace*{\fill} \pagebreak\end{center}
}
\def\beqn{\begin{eqnarray*}}
\def\eeqn{\end{eqnarray*}}
\def\beq{\begin{eqnarray}}
\def\eeq{\end{eqnarray}}
\def\boxeqn{\begin{boxequation*}}
\def\eoxeqn{\end{boxequation*}}
\def\boxeq{\begin{boxequation}}
\def\eoxeq{\end{boxequation}}
\def\grad{\nabla}
\def\bi{\begin{itemize}}
\def\ei{\end{itemize}}
\def\be{\begin{enumerate}}
\def\ee{\end{enumerate}}
\def\i{\item}

%
% ------ Vector defs  ------
%
\renewcommand{\mathbf}[1]{\mbox{\boldmath $ #1 $}}
\newcommand{\bvec}[1]{\mathbf{#1}}            % bold vector: \bvec{v}
\newcommand{\buvec}[1]{\mathbf{\hat{#1}}}     % bold unit vector
\newcommand{\uvec}[1]{\underline{#1}}         % underlined vector
\newcommand{\uuvec}[1]{\hat{\underline{#1}}}  % underlined unit vector
\newcommand{\zerobvec}{\mbox{\bf 0}}          % bold zero vector/matrix
\newcommand{\onebvec}{\mbox{\bf 1}}           % bold one vector/matrix
%
% 2 element column vector:
%                \twocvec{a}{b}  -->   ( a )
%                                      ( b )

\newcommand{\twocvec}[2]{\left(\begin{array}{c}
	#1 \\ #2
	\end{array}\right)}
\newcommand{\cvec}[1]{\left(\begin{array}{c}
	#1 
	\end{array}\right)}

\newcommand{\nchoosek}[2]{\twocvec{#1}{#2}}

%
% 2 element row vector:
%                \tworvec{a}{b}  -->   ( a  b )
\newcommand{\tworvec}[2]{\left(	{#1}\;\; {#2} \right)}
%
% indexed column vector:
%                \icvec{a}{1}{5}  -->   ( a_1 )
%                                       ( ... )
%                                       ( a_5 )
\newcommand{\icvec}[3]{\begin{array}{c}
	{#1}_{#2} \\ \vdots \\ {#1}_{#3}
	\end{array}}
%
% indexed row vector:
%                \irvec{a}{1}{5}  -->   ( a_1 ... a_5 )
\newcommand{\irvec}[3]{\begin{array}{ccc}
	{#1}_{#2} ,& \cdots &, {#1}_{#3}
	\end{array}}
%
% row vector:
%                \rvec{a}{1}{5}  -->   ( a_1 a_2 a_3 a_4 a_5 )
\newcount\m \newcount\n %
\def\rvec#1#2#3{ %
  \mbox{
    \m=#2 \n=#3%
    \loop $#1_{\number\m}$,
      \advance\m by 1 %
      \ifnum\m<\n %
    \repeat %
    $#1_{\number\m}$
  }
}
%
% ------ Matrix defs  ------
%
%
% diagonal matrix:
%      \diagmat{A}{B}  -->   ( A     0 )
%                            (   ...   )
%                            ( 0     B )
\newcommand{\diagmat}[2]{\left(\begin{array}{ccc}
#1 & & \mbox{\LARGE \bf 0} \\
 &\ddots &  \\
\mbox{\LARGE \bf 0} & &#2
\end{array}\right)}
%
% two by two  matrix:
%      \ttmat{a}{b}{c}{d}  -->   ( a b )
%                                ( c d )
\newcommand{\ttmat}[4]{\left(\begin{array}{cc}
#1 & #2 \\ #3 & #4
\end{array}\right)}



